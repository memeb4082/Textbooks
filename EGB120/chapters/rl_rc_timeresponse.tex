\chapter{RL and RC circuits and Time Response}
\label{cha:rl_rc_timeresponse}
    \section{Switches}
    \section{Natural Response}
        \subsection{Capacitors and Inductors}
            \begin{minipage}{0.5\linewidth}
                Capacitors store energy as voltage
                \begin{equation*}
                    i = C \odv{v}{t}
                \end{equation*}
            \end{minipage}
            \begin{minipage}{0.5\linewidth}
                Inductors store energy as current
                \begin{equation*}
                    v = L \odv{i}{t}
                \end{equation*}
            \end{minipage}
        \subsection{Switched RC Circuit}
            Assuming that the switch has been in the first position for a long time (till reached steady state).
            Find the voltage right before $t=0$.
            \begin{figure}[H]
                \centering
                \includegraphics[width=0.8\linewidth]{chapters/figures/switched_rc.png}
            \end{figure}
            For the initial condition
            \begin{itemize}
                \item Perform steady state analysis by treating the capacitor as an open circuit
                    \begin{figure}[H]
                        \centering
                        \includegraphics[width=0.4\linewidth]{chapters/figures/switched_rc_initial_cond.png}
                    \end{figure}
                    The voltage across the capacitor right before $t=0$ is the same as source voltage $V_s$
            \end{itemize}
            Natural Response of an RC Circuit
            \begin{figure}[H]
                \centering
                \includegraphics[width=0.6\linewidth]{chapters/figures/switched_rc_natural_resp.png}
            \end{figure}
            \begin{flalign*}
                \intertext{Using KCL we get}
                -C\odv{v}{t} &= \frac{v}{R_2} \\
                \intertext{Rearranging and solving the differential equation}
                \odv{v}{t} &= -\frac{v}{R_2 C} \\
                \frac{1}{v} \odv{v}{t} &= -\frac{1}{R_2 C} \\
                \intertext{Integrate both sides}
                \int \frac{1}{v} \odv{v}{t} \, dt &= \int -\frac{1}{R_2 C} \, dt \\
                \ln v &= -\frac{t}{R_2 C} + k \\
                v &= e^{-\frac{t}{R_2 C} + k} \\
                v &= Ae^{-\frac{t}{R_2 C}}\\
                \intertext{Noting that $A = v(0) = V_s$ we get}
                v &= V_s e^{-\frac{t}{R_2 C}}\\
            \end{flalign*}
        \subsection{Switched RL Circuit}
            Assuming that the switch has been in the first position for a long time (till reached steady state).
            Find the current right before $t=0$.
            \begin{figure}[H]
                \centering
                \includegraphics[width=0.8\linewidth]{chapters/figures/switched_rl.png}
            \end{figure}
            For the initial condition
            \begin{itemize}
                \item Perform steady state analysis by treating the inductor as a short circuit
                    \begin{figure}[H]
                        \centering
                        \includegraphics[width=0.4\linewidth]{chapters/figures/switched_rl_initial_cond.png}
                    \end{figure}
                    The current through the inductor right before $t=0$ is the same as source current $I_s$
            \end{itemize}
            Natural Response of an RL Circuit
            \begin{figure}[H]
                \centering
                \includegraphics[width=0.3\linewidth]{chapters/figures/switched_rl_natural_resp.png}
            \end{figure}
            \begin{flalign*}
                \intertext{Using KVL we get}
                -L\odv{i}{t} &= R_2 i \\
                \intertext{Rearranging and solving the differential equation}
                \odv{i}{t} &= -\frac{R_2}{L} i
                \intertext{Using the integrating factor method}
                \odv{i}{t} + \frac{R_2}{L} i &= 0 \\
                \intertext{Multiplying both sides by $e^{\frac{R_2}{L} t}$}
                e^{\frac{R_2}{L} t} \odv{i}{t} + \frac{R_2}{L} e^{\frac{R_2}{L} t} i &= 0 \\
                \intertext{Noting that $\odv{}{t} \left( e^{\frac{R_2}{L} t} i \right) = e^{\frac{R_2}{L} t} \odv{i}{t} + \frac{R_2}{L} e^{\frac{R_2}{L} t} i$}
                \odv{}{t} \left( e^{\frac{R_2}{L} t} i \right) &= 0 \\
                \intertext{Integrating both sides}
                \int \odv{}{t} \left( e^{\frac{R_2}{L} t} i \right) \, dt &= \int 0 \, dt \\
                e^{\frac{R_2}{L} t} i &= k \\
                i &= ke^{-\frac{R_2}{L} t} \\
                \intertext{Noting that $k = i(0) = I_s$ we get}
                i &= I_s e^{-\frac{R_2}{L} t} \\
            \end{flalign*}
        \subsection{Natural Response}
            \begin{equation*}
                v(t) = v(0)e^{-\frac{t}{R_2 C}} \tag{Where $\tau = RC$ is the time constant}
            \end{equation*}
            \begin{figure}[H]
                \centering
                \includegraphics[width=0.6\linewidth]{chapters/figures/natural_response_rc.png}
                \caption{Natural Response of RC Circuit for $v(t) = v(0)e^{-\frac{t}{R_2 C}}$ where $v(0) = 1V$, $R_2 C = 1\Omega$, and $C = 1F$} 
            \end{figure}
            \begin{equation*}
                i(t) = i(0)e^{-\frac{t}{R_2 C}} \tag{Where $\tau = \frac{L}{R_2}$ is the time constant}
            \end{equation*}
            \begin{figure}[H]
                \centering
                \includegraphics[width=0.6\linewidth]{chapters/figures/natural_response_rl.png}
                \caption{Natural Response of RL Circuit for $i(t) = i(0)e^{-\frac{t}{R_2 C}}$ where $i(0) = 1A$, $R_2 C = 1\Omega$, and $L = 1H$}
            \end{figure}
    \section{Step Response}
