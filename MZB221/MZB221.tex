\documentclass{book}
\usepackage{glossaries}
\usepackage{derivative}
\usepackage{geometry}
\usepackage{pgfplots}
\usepackage{graphicx}
\usepackage{tikz}
\usepackage{float}
\usepackage{xcolor}
\usepackage{amsmath}
\usepackage{amssymb}
\geometry{
    left=10mm,
    right=10mm,
    top=10mm,
    bottom=10mm
}
\begin{document}
    \title{
        % \begin{flushleft}
            Queensland University of Technology\\
            \rule{\linewidth}{0.5pt}
        % \end{flushleft}
        \centering
        \textbf{MZB221} \\
        Electrical Engineering Mathematics\\
        \vspace{0.4cm}
        % \large{BASA MARS-242 Message Transmission}
        \rule{\linewidth}{1.5pt}
        \small{\textit{Professor Nicholas Buttle}}
    }
    \author{Dinal Atapattu}
    \date{\today}
    \maketitle
    \thispagestyle{empty}
    \tableofcontents
    \chapter{Infinite Series}
        \section{Sequences, Infinite Series, Convergence}
        \section{Taylor polynomials, Taylor series, Radius of convergence}
        \section{Introduction to Fourier series}
        \section{Constructing Fourier series}
    \chapter{Vector Calculus}
        \section{Introduction to Vector Calculus, div, grad, curl}
            \subsection{Scalar Fields}
                A scalar field is a function
                \begin{flalign*}
                    f:\mathrm{R}^n \to \mathrm{R}
                \end{flalign*}
                \textcolor{red}{\fbox{$n=2$}} $f=f(x,y)$
                \begin{itemize}
                    \item 2 independent variables $x,y$
                    \item $f$ is a function that has $x$ and $y$ as inputs and a single real number as
                    the output
                \end{itemize}
                Phyiscal examples of scalar fields are 
                \begin{itemize}
                    \item The temperature $T(x,y,z)$, the pressure $p(x,y,z)$,
                    the density $\rho (x,y,z)$ of a fluid
                    \item Concentration of a pollutant in a lake $c(x,y,z)$
                    \item Height of a surface or a mountain $h(x,y)$
                    \item Charge density $\rho (x,y,z)$, electrical potential $V(x,y,z)$
                \end{itemize}
                \subsubsection*{Partial Derivatives}
                    For $f:\mathrm{R}^2 \to \mathrm{R}$
                    \begin{flalign*}
                        \pdv{f}{x} = \lim_{h\to 0} \frac{f(x+h,y)-f(x,y)}{h}, \pdv{f}{y}=\lim_{h\to 0} \frac{f(x,y+h)-f(x,y)}{h}
                    \end{flalign*}
                    \begin{itemize}
                        \item $f_x$ is the rate of change of $f$ in the $x$-direction (y is constant) (and vice-versa)
                        \item Geometrically, $f_x$ and $f_y$ are the slopes of the surface $z=f(x,y)$ in the $x$ and $y$ directions.
                    \end{itemize}
                    For $f:\mathrm{R}^3 \to \mathrm{R}$
                    \begin{flalign*}
                        \pdv{f}{x} = \lim_{h\to 0} \frac{f(x+h,y)-f(x,y)}{h}&, \pdv{f}{y}=\lim_{h\to 0} \frac{f(x,y+h)-f(x,y)}{h}\\
                        \pdv{f}{z} \lim_{h\to 0} \frac{f(z+h,y)-f(z,y)}{h}
                    \end{flalign*}
                    \textcolor{red}{No equivalent geometric representation as "surface" ($f(x,y,z)$) is a 4-dimensional hypersolid}
                \subsubsection*{Directional Derivative of a Scalar Field}
        \section{Review of Multiple Integration, Change of Variables}
        \section{Introduction to cylindrical and spheroidal coordinates, integration}
        \section{Line Integrals, Surface Integrals}
    \chapter{Differential Equations}
        \section{Introduction to Laplace transform, Strategy for Solving Linear ODEs}
            \subsection{Definition of the Laplace Transform}
                The Laplace Transform of the function $f(t)$ is defined as
                \begin{flalign*}
                    \mathcal{L}\{f(t)\} = \int^{\infty}_{0} e^{-st}f(t)dt
                \end{flalign*}
                for values of $s$ that the integral converges\\
                The notation $F(s)$ is often used to represent $\mathcal{L}\{f(t)\}$
        \section{Further Properties of Laplace Transforms, solving more complicated initial value problems}
        \section{Non-Linear first-order ODEs, Phase lines, Stability, Bi-Furcation}
        \section{Linear Systems of ODEs, Exact Solutions, Classification, Non-Homogeneous Systems}
        \section{Non-Linear Systems of ODEs, Phase Plane, Nullclines, Stability}
    \chapter{Practice Exams}
    Show that $f(x) = \frac{1}{x-2}$ about $x=3$ is 
    \begin{flalign*}
        f(x) &= \sum^{\infty}_{n=0} \left(-1\right)^n \left(x-3\right)^n, 3-R < x < 3+R\\
    \end{flalign*}
    \begin{align*}
        &f'(x) = \frac{1}{(x-2)^2} &&f''(x) = \frac{-2}{(x-2)^3} &&&f'''(x) = \frac{6}{(x-2)^4}\\
        &f'(3) = \frac{1}{(3-2)^2} &&f''(3) = \frac{-2}{(3-2)^3} &&&f'''(3) = \frac{6}{(3-2)^4}\\
        &f'(3) = 1 &&f''(3) = -2 &&&f'''(3) = 6\\
    \end{align*}
    \begin{flalign*}
        f(x) &= \sum^{\infty}_{n=0} \frac{f^{(n)}(a)}{n!}(x-a)^n\\
        \intertext{For}
    \end{flalign*}
\end{document}